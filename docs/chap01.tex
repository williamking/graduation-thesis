%% chapter 1 Introduction
\chapter{引言}
%定义,过去的研究和现在的研究,意义,与图像分割的不同,going deeper
2016年,Matrix在线评测系统在中山大学软件学院诞生,并迅速为学院各大程序课程
作业、考试所应用。Matrix支持编程题、选择题、简答题等题型的评测或批改。随着Matrix
系统的扩展,支持新的题型也成了Matrix评测系统的一个新需求。本文致力于程序填空题的
评测算法设计以及在系统中的部署工作,为Matrix评测系统增加新的程序填空题题型的评测
功能。

本文分为四节,首先阐述程序填空题评测问题背景和意义,然后介绍国内外相关研究背景与现状,
接着介绍本文的大体工作,最后介绍本文的结构。
\label{cha:introduction}
\section{程序填空评测问题的背景和意义}
\label{sec:background}
% What is the problem
% why is it interesting and important
% Why is it hards, why do naive approaches fails
% why hasn't it been solved before
% what are the key components of my approach and results, also include any specific limitations,do not repeat the abstract
%contribution
程序填空题在程序设计课程考试中是一种重要的考察题型,学生通过阅读分析缺失部分的代码与题干内容,
理解题目的意图和程序的思路,填写出程序中缺失的部分。
由于程序的语法的程序填空题型的答案一般较为固定,一般为一条语句或者表达式,主观性较弱,
因此具有易批量自动批改性。
如今大学课程的程序填空题多出现于笔试中,由人工进行批改,效率较低。
学院Matrix在线评测系统的上线,使得程序填空题的自动评测成为了可能和需求。
在程序填空题自动评测实现的环境下,学生通过电脑提交填空答案,由评测服务器进行自动评测得出评分,
有效提高批改效率以及准确度。对于Matrix评测系统的功能扩展以及程序填空题的应用算法研究具有
重大的意义。
\section{研究背景和现状}
\label{src:related_work}
程序填空题是一种语言性算法的应用场景。程序填空题的特征在于其题干主要部分为程序代码,答案为
程序语言,并且程序代码具有可执行性。因此与程序填空题相关算法的研究包括程序语言的研究以及
一般文本语言的研究(应用于普通填空题)。
\subsection{填空题相关领域的研究}
\label{src:program_filling_related_work}
对于一般的文本填空题,可看作为多道简答题的集合(每个空为一道简答题)。简答题的一般评分方式为将答案与标准答案
进行语义性比对或者字符相似性比对。
因此可以着眼于基本的字符串相似性度量和
从答案的语义性进行分析的相似性度量的相关性研究。

Andres Marzal和Enrique Vidal(1993)提出了正则化的编辑距离计算算法和应用\cite{Marzal1993Computation}。
这种相似性度量算法是语义无关的。
RadaMihalcea等(2006)提出了给予语料库和知识库的文本相似度的度量,
将相似性的度量上升到了语义的级别\cite{Mihalcea2006Corpus}。
Xiaoling Wang等(2007)提出了基于Gram框架的字符串相似度搜索算法\cite{DBLP:journals/tkde/HuZWZ15_20}。
Marios Hadjieleftheriou等(2014)提出了以编辑距离算法为基础的
使用B+树的字符串相似性搜索\cite{lu2014efficiently_19}。
Lorraine A. K. Ayad等(2016)提出固定长度近似字符串匹配软件库的开发\cite{DBLP:journals/bmcbi/AyadPR16_18}。
\subsection{程序语言相关的研究}
程序语言特征方面的研究也是对程序填空题的自动评测的一种可参考的领域。
David T.Barnard等(1982)提出了利用LR文法修复语法错误的方法\cite{Barnard1982Hierarchic}。
Naohiro Ishiid等(1996)提出了移除源代码多样性的方法\cite{Hattori1996A}。
Cliare Le Goues等(2012)提出了自动修复程序出现bug的方法\cite{Goues2012A}。
Anyu Wang等(2015)提出了局部修复码(Locally Repairable Codes)的基于程序的整形边界\cite{DBLP:journals/tit/WangZ15_22}。
Paul W. McBurney等(2016)提出了对于源代码间文本相似度的经验性学习算法\cite{DBLP:journals/ese/McBurneyM16_23}。
Dave Towey等(2017)提出了一种可变的支持无测试oracle的程序修复的测试方法。\cite{DBLP:journals/jss/JiangCKTD17_21}。
针对程序语言进行分析可以从语句中获取关键的信息,是对信息的另一种过滤,对于提高程序填空评测的准确度有着一定的意义。
\section{本文的工作}
本文主要工作分为以下三个部分:
\begin{enumerate}
  \item 设计程序评测的算法及对应的交互流程。
  算法采用动态测试与静态分析相结合的算法。
  \item 将算法部署到 Matrix 应用系统中。
  本算法针对服务端和评测系统的代码逻辑,将程序填空题型的评测算法在服务端、评测系统及之间的
  交互中实现。
  \item 对算法效果进行实验和改进。本文通过评测算法的初步应用,
  找出算法之中存在的缺陷,并设计改进算法的方案。
  通过相同的答案提交数据进行对比实验,分析出改进算法的评测效果的准确率提高以及提高原因。
\end{enumerate}
\section{本文的论文结构与章节安排}
\label{sec:arrangement}
本文共分为六章,各章节内容安排如下:

第一章引言。

第二章介绍了本文相关的领域的一些选型算法知识,
分为针对代码可执行性的动态测试以及针对语言及内容特性的静态分析。

第三章阐述了本文的主要工作,包括算法的选择和系统各部分的部署以及实验后
算法出现的问题。

第四章阐述了针对算法问题的改进以及为改进做的工作。

第五章阐述了改进后的实验结果与分析。

第六章总结了这次工作的不足与后续工作的展望。
\section{本章小结}
本章介绍了本文研究的程序填空问题的意义以及相关研究背景、本文的工作以及论文结构,
下一章会阐述程序填空评测算法设计所参考的各种选型算法以及基本概念。

\clearpage
