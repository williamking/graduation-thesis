\chapter{动态测试与静态分析结合的评测算法的实现与部署}
\label{cha:assumption}
\subsection{工作总结}
本文结合程序评测与文本相似度算法用于程序填空题型的评测,是一种新的尝试。
尤其在考试试验后对程序评测部分做的进一步改进大大提高了评测的效果。与他人在
相关领域的工作相比,本文由于时间、成本、应用需求等原因,没有采用语义分析、
语法修复等算法进行实验,这是一点遗憾和缺陷之处。但本文在算法与工程的结合方
面作出了一定的努力,这是一点可取之处。

\subsection{系统的不足和改进方向}
在程序填空算法应用到 Matrix 应用系统的工作中,职责的分离度仍不够高。比
如改进后的动态测试评测算法中,对一道题需要向评测系统发起多次的提交评测,这
是比较不合理的。因为这属于评测逻辑的一部分,按照职责粒度的划分,应该有评测
系统收到评测请求后运行多次程序计算出动态测试评测部分的总得分,而不应该由服
务端去进行多次评测请求和发送。然而无法这样做的原因是评测系统本身内部的逻
辑结构难以改动、改动成本高以及和评测系统开发者的对接沟通等问题。

另一方面,静态分析部分使用编辑距离算法计算相似度仍有缺陷,一是编辑距离算法
是语义无关的算法,无法抓住答案的关键得分点在哪里,一些无关紧要的相似部分也
会被编辑距离算法所检测而算入得分(如括号,分号,运算符等非重点或无意义的字
符)。要想在静态分析部分的评测效率更近一步,如何设计语义化的评测以及配合语
义化的评测算法的交互流程是关键。

\clearpage
